\documentclass[12pt]{article}
\usepackage{xltxtra} 


% \usepackage{avant}
% \usepackage{bookman}
% \usepackage{helvet}  %chybí tuèný øez u strojového písma
% \usepackage{newcent}
% \usepackage{palatino}

\renewcommand{\figurename}{Figure}

\usepackage[labelsep=endash]{caption}

\usepackage[bookmarksopen,colorlinks,plainpages=false,urlcolor=blue]{hyperref}
\usepackage{url}
\usepackage{ifpdf}
\usepackage{indentfirst}

% \ifthenelse{\isundefined{\pdfoutput}}
\ifpdf
  \usepackage[dvipdf]{graphicx}
\else
  \usepackage[dvips]{graphicx}
\fi

\usepackage[top=2cm, left=2cm, text={17cm, 26cm}, ignorefoot]{geometry}

\begin{document}
\begin{titlepage}

% \vspace*{1cm}
\begin{center}
\begin{LARGE}
VYSOKÉ UČENÍ TECHNICKÉ V BRNĚ \\


\end{LARGE}
\begin{Large}
Fakulta informačních technologií
\end{Large}
\end{center}

\vfill
\vspace{4cm}
\begin{center}
\begin{Large}
Bezpečnost informačních systémů \\
\end{Large}
\bigskip
\begin{Huge}
Program pro detekci spamu v e-mailové komunikaci\\
\end{Huge}
\end{center}

\vfill

\begin{center}
\begin{Large}
\today
\end{Large}
\end{center}

\vfill

\begin{flushleft}
\begin{large}
\begin{tabular}{ll}
Autor: & Jan Pawlus, \url{xpawlu00@stud.fit.vutbr.cz} \\ 
\end{tabular}
\end{large}
\end{flushleft}
\end{titlepage}

\newpage

\section{Princip funkčnosti} \pagestyle{myheadings}\pagenumbering{arabic}\setcounter{page}{1}

Můj e-mailový filtr je složen ze tří modulů - \verb|bis.py|, kde je spuštěna logika, \verb|ArgParser| (\verb|argParser.py|), kde probíhá zpracovávání jednotlivých argumentů (e-mailů) a parsování, a \verb|EmailAnalyzer| (\verb|analyze.py|), kde probíhá analýza a aplikování regulárních výrazů na jeden e-mail. Projekt jsem implementoval v jazyce \textit{python3}.

\section{ArgParser}

V tomto modulu probíhá zpracování jednotlivých e-mailů poskytnutých od uživatele jako argumenty skriptu. Pokud je argument zadán správně a soubor se podaří načíst (v opačném případě skončí analýza výstupem \verb|FAIL|), je využita knihovna \textit{eml-parser}, která e-mail zpracuje vstupní soubor do lepší struktury (\verb|dictionary|). Každý e-mail ještě skript upraví tak, že substituuje diakritiku standardními znaky. Tento převod zajišťuje knihovna \textit{unidecode}. Díky tomu je možné lépe analyzovat české či slovenské e-maily.

\section{EmailAnalyzer}

Analýza jednotlivých e-mailů funguje na principu aplikování několika regulárních výrazů rozdělených dle závažnosti a významu na e-mail. Každý takový regulární výraz tedy obsahuje významově podobná slova a má přiřazen určitý koeficient, neboli skóre. Pokud tedy některý z regulárních výrazů uspěje, přičte se k celkovému skóre e-mailu tento koeficient (znásobený počtem výskytu klíčových slov daného regulárního výrazu). Vyhodnocení pak funguje jako porovnání celkového skóre s určitým prahem - pokud je skóre vyšší, než práh, e-mail je označen jako spam.

\subsection{Příklady aplikovaných významových regulárních výrazů}

Aby bylo lépe pochopitelné, jak skript funguje, je třeba se podívat na určitý regulární výraz. Například detekce klíčových slov spojené se sexem je rozděleno do tří regulárních výrazů. V prvním, které má nejvyšší koeficient, jsou obsaženy slova jako \verb|naked photos| nebo \verb|suck your dick|. Pokud jsou taková slova v e-mailu nalezena, je v podstatě rovnou označen jako spam.

Ve druhém regulárním výrazu, který již při nalezení dává e-mailu nižší skóre, tedy k označení jako spam je nutné nalézt alespoň dvě taková slova, jsou obsaženy fráze jako \verb|erotic|, \verb|premature ejaculate| nebo \verb|enlarging oil|. V nejméně závažném fráze jako \verb|your husband| nebo \verb|your wife|. Tímto je zlepšena ochrana před označením validního e-mailu jako \verb|SPAM|, pokud se v e-mailu vyskytne méně závažná fráze.

Podobným stylem jsou ošetřeny i další typy nevyžádané pošty, jako léky, dědictví, výhry v loteriích, snadné triky k vydělání peněz a jiné. Tyto okruhy a klíčová slova jsem dal dohromady pouze pocitově z osobní analýzy různých datových setů se spamem. Většinu klíčových slov jsem se snažil také překládat do češtiny či slovenštiny, aby skript dokázal detekovat také náš spam.

\newpage

\subsection{Detekce klíčových slov bez významu}

Tento skript nepracuje pouze s hledáním významových frází - snažil jsem se také aplikovat některá z pravidel z projektu \verb|SpamAssassin|. Kupříkladu pokud doména odesilatele obsahuje slova jako \verb|offer|, skóre e-mailu narůstá. Dalším příkladem může být například \textit{HTML} odkaz, který odkazuje na určitou IP adresu (využívané například u \textit{phishingu}), nebo pokud e-mail má pouze \textit{HTML} část, ale ne \textit{plaintext}. Skript také přičítá malé skóre, pokud e-mail obsahuje větší množství slov napsaného v kapitálkách. 

\section{Výstup a použití}

Skript obsahuje soubor \verb|Makefile|, který vytvoří symbolický odkaz. Poté je možno s programem pracovat jako se spustitelným souborem \verb|antispam|. Skript ověří všechny e-maily zadané argumenty. Pokud soubor neexistuje, vypíše k němu skript \verb|FAIL|, následně proběhne analýza, kdy skript k zadanému e-mailu vypíše \verb|SPAM| či \verb|OK|. Pokud je e-mail označen jako \verb|SPAM|, jsou vypsána i klíčová slova, na základě kterých skript e-mail jako \verb|SPAM| vyhodnotil (díky formě aplikování regulárního výrazu funkcí \verb|findall| má skript přístup k těmto klíčovým slovům).


\end{document}